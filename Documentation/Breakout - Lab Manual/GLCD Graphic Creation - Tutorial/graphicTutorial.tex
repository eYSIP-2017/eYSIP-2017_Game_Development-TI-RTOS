\documentclass{article}
\usepackage[utf8]{inputenc}
\usepackage{graphicx}
\usepackage{parskip}
\usepackage{textcomp}
\usepackage{hyperref}
\hypersetup{
    colorlinks=true,
    linkcolor=blue,
    filecolor=magenta,      
    urlcolor=cyan,
}
\usepackage{listings}
\lstset{language=C,
                basicstyle=\ttfamily,
                keywordstyle=\color{blue}\ttfamily,
                stringstyle=\color{red}\ttfamily,
                commentstyle=\color{green}\ttfamily,
                morecomment=[l][\color{magenta}]{\#}
}

\title{The Vending Machine Controller - RTOS}
\author{By \\ Umang Deshpande and Akshay Hegde}
\date{June 2017}

\begin{document}
\maketitle

\section{Software used}
\qquad \textit{Mikroelectronika GLCD Font Creator} is used for this purpose. GLCD Font Creator enables the creation of personalized fonts, symbols and icons for LCDs and GLCDs. Create fonts and symbols from scratch, or by importing existing fonts on your system. It lets you modify and adjust them for your needs, apply effects and finally export them as source code for use in mikroC, mikroBasic or mikroPascal compilers.

\subsection{Downloading and Installing Mikroelektronika GLCD Font Creator}
Download the Windows Installer from their \href{https://download.mikroe.com/setups/additional-software/glcd-font-creator/glcd-font-creator-v120.zip}{official website}. The Install Wizard is pretty straightforward. \\
For macOS, download the same .exe. Then use Wine to open it(Tutorials on installing and using Wine on macOS can be found \href{https://www.davidbaumgold.com/tutorials/wine-mac/}{here}). 

\section{Creating a font library}

\end{document}
